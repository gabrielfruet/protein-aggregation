\documentclass[12pt]{extarticle}

\usepackage{sectsty}
\usepackage{graphicx}
\usepackage[portuges]{babel}

% Margins
\topmargin=-0.45in
\evensidemargin=0in
\oddsidemargin=0in
\textwidth=6.5in
\textheight=9.0in
\headsep=0.25in

\title{\textbf{Projetando uma Insulina Mais Estável: Algoritmos Genéticos e Dinâmica Molecular Aplicados à Engenharia de Proteínas}}
\author{\textbf{Gabriel Fruet, Mateus Ribeiro }}
\date{\today}

\begin{document}
\maketitle	

\section{Introdução}

A insulina é um hormônio essencial no controle da glicemia e seu uso
terapêutico é indispensável para milhões de pessoas com diabetes ao redor do
mundo. No entanto, a estabilidade térmica da insulina representa um desafio
significativo, especialmente em regiões de clima quente ou com acesso limitado
à refrigeração. A degradação da insulina em temperaturas elevadas compromete
sua eficácia, exigindo soluções que aumentem sua estabilidade sem comprometer
sua funcionalidade biológica.  

Os algoritmos genéticos (AGs) têm se destacado como uma abordagem promissora
para problemas de otimização em bioinformática e engenharia de proteínas.
Inspirados na seleção natural, esses algoritmos simulam processos evolutivos
para encontrar soluções otimizadas dentro de um espaço de busca complexo. No
contexto da engenharia de proteínas, os AGs podem ser empregados para
identificar mutações que conferem maior estabilidade térmica à insulina,
preservando sua atividade biológica.  

Este trabalho explora a aplicação de algoritmos genéticos na identificação de
mutações que aumentem a termoestabilidade da insulina. Para isso, utilizamos
modelos computacionais para simular mutações e avaliar seu impacto na
estabilidade da proteína. A metodologia proposta visa acelerar o
desenvolvimento de variantes mais estáveis da insulina, contribuindo para
avanços na terapia do diabetes e na biotecnologia farmacêutica.  

\section{Materiais e Métodos}

\subsection{Bibliotecas e Tecnologias}

Os experimentos foram realizados em computador com uma NVIDIA Geforce RTX 3060
com 12GB de VRAM e um processador Ryzen 5 5600, para aceleração
das predições estruturais via ESMFold, foi utilizado CUDA. O pipeline foi implementado em Python,
integrando as bibliotecas PyGAD, PyTorch, BioPython, OpenMM e outras dependências.

\subsection{População Inicial}

A sequência primária da insulina foi utilizada como ponto de partida para a
geração de variantes mutantes. Cada indivíduo na população inicial corresponde
a uma variante da insulina onde apenas um único aminoácido é alterado. As
sequências geradas são representadas em formato FASTA para facilitar a
manipulação computacional.

\subsection{Cálculo da Estrutura Conformacional}

Ao trabalharmos com proteínas, por vezes se faz necessário ter o conhecimento da sua
estrutura conformacional. Porém, o conhecimento prévio desse dado nem sempre está disponível,
especialmente quando trabalhos com proteínas mutantes. 

Atualmente, temos diversos modelos computacionais que conseguem prever a
estrutura conformacional de uma proteína, tais como AlphaFold \cite{alphafold},
OmegaFold \cite{omegafold}, RoseTTAFold \cite{rosettafold}, ESMFold
\cite{esmfold}. No entanto, a maioria desses modelos exigem um grande poder
computacional, como o AlphaFold, que exige em torno de 80GB de memoria VRAM,
além de disco rígido na ordem de terabytes para armazenar os banco de dados de
proteínas usados no \textit{multiple sequence alignments} (MSA).

Com esse problema em vista, o ESMFold foi proposto. O principal diferencial do
ESMFold é a quantidade menor de parâmetros, no nosso caso, 3 bilhões,
necessitando de um hardware menos potente e a predição sem utilizar MSA. Isso
faz com que o ESMFold consiga ser executado na ordem de poucos segundos em GPUs
de entrada.

\subsection{Cálculo da Termoestabilidade}

A termoestabilidade de proteínas pode ser avaliada por métodos como a energia
livre de dobramento $\Delta G$, a temperatura de desnaturação ($T_m$), a flutuação
quadrática média dos átomos (RMSF) e a fração de contatos nativos (QQ). A
escolha do método depende do sistema estudado, dados experimentais e objetivos
da análise.

A $\Delta G$ é uma medida direta da estabilidade, mas requer simulações avançadas. O $T_m$
é utilizado em estudos experimentais, mas precisa de simulações em várias
temperaturas. A RMSF mede a flexibilidade local, mas não a estabilidade global.

Optamos pela fração de contatos nativos (QQ) devido à sua simplicidade,
eficiência e capacidade de quantificar a preservação da estrutura nativa em
simulações de dinâmica molecular. QQ reflete a proporção de contatos atômicos
mantidos em relação à estrutura nativa, sendo robusta a variações nas condições
de simulação.

O cálculo de QQ foi feito com a biblioteca MDTraj \cite{McGibbon2015},
utilizando a estrutura nativa como referência e o esquema "closest-heavy" para
contatos atômicos. Três réplicas independentes foram simuladas para cada
sistema, com os valores de QQ reportados como média ± desvio padrão. As
simulações usaram o campo de forças AMBER14 \cite{Maier2015} e o modelo de água
TIP3P \cite{Jorgensen1983}.

\end{document}
